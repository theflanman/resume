%%%%%%%%%%%%%%%%%%%%%%%%%%%%%%%%%%%%%%%
% Deedy CV/Resume
% XeLaTeX Template
% Version 1.0 (5/5/2014)
%
% This template has been downloaded from:
% http://www.LaTeXTemplates.com
%
% Original author:
% Debarghya Das (http://www.debarghyadas.com)
% With extensive modifications by:
% Vel (vel@latextemplates.com)
%
% License:
% CC BY-NC-SA 3.0 (http://creativecommons.org/licenses/by-nc-sa/3.0/)
%
% Important notes:
% This template needs to be compiled with XeLaTeX.
%
%%%%%%%%%%%%%%%%%%%%%%%%%%%%%%%%%%%%%%

\documentclass[letterpaper]{deedy-resume} % Use US Letter paper, change to a4paper for A4 
\usepackage{multicol}

\begin{document}

%----------------------------------------------------------------------------------------
%	TITLE SECTION
%----------------------------------------------------------------------------------------

\lastupdated % Print the Last Updated text at the top right

\namesection{Connor}{Flanigan}{ % Your name
% \urlstyle{same}\url{https://flanigan.engineering/} \\ % Your website, LinkedIn profile or other web address  % commenting out until I fix the DNS records
\href{mailto:connorflanigan@gmail.com}{connorflanigan@gmail.com} | (207) 332-0210 \\  % Your contact information
Github:// \href{https://github.com/CFlaniganMide}{\bf CFlaniganMide} \\
Github:// \href{https://github.com/theflanman}{\bf TheFlanMan} \\
LinkedIn:// \href{https://www.linkedin.com/in/connor-flanigan}{\bf Connor-Flanigan}
}

%------------------------------------------------
% Education
%------------------------------------------------

\section{Education} 

\subsection{Worcester Polytechnic Institute \hfill}

\descript{BS in Robotics Engineering}
\location{October 2016 | Worcester, MA}

%------------------------------------------------
% Skills
%------------------------------------------------
%
% \section{Skills}
% 
% \subsection{Languages}
%
% \location{Significant Experience:}
% Python \textbullet{} Matlab \textbullet{} Lua \\ 
% \location{Some Experience:}
% Java \textbullet{} JavaScript \textbullet{} HTML \\
% CSS \textbullet{} C/C++ \textbullet{} Labview \textbullet{} C\# \\
%
% \subsection{Development Activities}
% Software Design \textbullet{} Requirements Elicitation
%
% \subsection{Packages and Toolkits}
% \location{Python:}
% NumPy \textbullet{} SciPy \textbullet{} TensorFlow \textbullet{} ROS
% \location{Matlab:}
% Signal Processing \textbullet{} Compiler \textbullet{} MEX
%
% \subsection{Analysis}
% Signals Processing \textbullet{} Spectral Analysis \textbullet{} Fourier Analysis \textbullet{} Circuits \textbullet{} Software Profiling
%
% \subsection{Software}
% Solidworks \textbullet{} ESPRIT \textbullet{} Pycharm
%
% \sectionspace % Some whitespace after the section

%------------------------------------------------
% Experience
%------------------------------------------------

\section{Professional Experience}

\runsubsection{Midé Technology}
\descript{| Software Engineer II}

\location{November 2017 - Present | Woburn, MA}
\begin{tightitemize}
\item Created a hardware/firmware testing system using Raspberry Pis and Github Actions.  This required Python scripts, systemd services, udev rules, and a custom Ubuntu image.
\item Built DeMCAS, a tool to optimize parametric aerofoil designs in Matlab.  This primarily utilized a gradient descent method to reach an optimal solution.
\item Technical lead for the Endaq Analyzer, a Matlab tool for visualizing and analyzing timeseries data using spectral techniques such as spectrograms and PSDs.
\item Developed a library in Python to classify rat calls using spectral analysis, image recognition methods, and bayesian estimation.  The library went through multiple revisions, utilizing Numpy, SciPy, Scikit-Learn, and Tensorflow.
\item Wrote SBIR contract proposals focussing on spectrographic signal analysis, machine learning, and condition-based maintenance.
\item Developer for Midé's open source Python libraries, which are used to analyze shock and vibration data.
\end{tightitemize}

\sectionspace % Some whitespace after the section

%------------------------------------------------

\runsubsection{IDEXX}
\descript{| Software Engineer}

\location{February 2017 – October 2017 | Westbrook, ME}
\begin{tightitemize}
\item Developing an interface for running custom chemical assays using IDEXX's Catalyst machines.  The client frontend used JavaScript to create dynamic forms, and sent web requests to the device's server backend written in Lua.
\item Created a test rig to replicate manufacturing errors for root cause analysis.  The software control was written in Labview, and ran on a CRIO FPGA.
\item Designed experiments to identify errors in automated immuno-assay tests.  Utilized high speed camera and machine vision, written in Labview and Python.
\item Created electrical breakouts to interface National Instruments hardware and our custom test rigs.
\end{tightitemize}

\sectionspace % Some whitespace after the section

%------------------------------------------------

\runsubsection{WPI Robotics Lab}
\descript{| Lab Designer, TA}

\location{July 2015 - September 2015 | Worcester, MA}
\begin{tightitemize}
\item Wrote labs for WPI's mobile robotics course, in order to create a more coherent experience for students.  The labs are written in Python and utilize ROS to control Turtlebot 2's; the final lab has students use their prior labs to control the turtlebot in order to map an environment using SLAM.
\end{tightitemize}

\sectionspace % Some whitespace after the section

%------------------------------------------------

% \runsubsection{Wasco Products}
% \descript{| Engineering Intern}
%
% \location{July 2013 - September 2013, June 2014 - September 2014 | Wells, ME}
% \begin{tightitemize}
% \item Developed software in C\# for integration into a server to automate designs for a variety of products within a team of three developers.
% \item Wrote instruction manuals for assembly and installation of Wasco's residential skylights.
% \end{tightitemize}
%
% \sectionspace % Some whitespace after the section

% \end{minipage} 



%------------------------------------------------
% Projects
%------------------------------------------------

% \newpage % Start a new page

\section{Projects}

\runsubsection{Midé Technology}
\descript{| \underline{RASP} \href{https://www.sbir.gov/node/1627655}{(Phase I)} \href{https://www.sbir.gov/node/1588569}{(Phase II)}}
\location{December 2017 - August 2021}
RASP is an audio processing library developed in Python, utilizing Numpy, SciPy, Scikit-Learn, and Tensorflow across it's different iterations.  The first phase of the contract was largely self-directed, and required me to refine my skills in signal processing and image analysis.  In phase II, we refined the system with a few methods, namely bayesian estimation and recurrent neural networks.

\sectionspace % Some whitespace after the section

%------------------------------------------------

\runsubsection{Midé Technology}
\descript{| \href{https://github.com/MideTechnology/ebmlite}{\underline{ebmlite}} | \href{https://github.com/MideTechnology/idelib}{\underline{idelib}} | \href{https://github.com/MideTechnology/endaq-python}{\underline{endaq-python}}}
\location{Ongoing open-source development}
I'm an active developer on all of Midé's open-source libraries, which are all developed in python and focus on analyzing shock and vibration data.  The analysis backend is built on Numpy, Pandas, and SciPy, with a plotting front-end built on Plotly.  In addition to the library itself, I also developed a CI/CD pipeline for these libraries using Github Actions.

\sectionspace % Some whitespace after the section

%------------------------------------------------

\runsubsection{Midé Technology}
\descript{| \href{https://info.mide.com/data-loggers/slam-stick-lab-software}{\underline{Endaq Analyzer}}}

\location{February 2018 - present}
The Endaq Analyzer is a standalone program developed in Matlab to analyze shock and vibration data.  The is built with the Matlab Compiler and makes heavy use of the Signal Processing Toolbox to generate PSD's, spectrograms, waterfall plots, and moving metrics.

\sectionspace % Some whitespace after the section

%------------------------------------------------

\runsubsection{Midé Technology}
\descript{| \href{https://www.sbir.gov/node/1485253}{\underline{DeMCAS}}}

\location{April 2018 - December 2019}
DeMCAS is a Matlab-based software tool which uses gradient descent to optimize the size and profile of aerofoils.  In addition to the codebase, I was also responsible for creating physical test rigs to validate the mechanical, electrical, and aerodynamic perfomance of our simulations.

\sectionspace % Some whitespace after the section

%------------------------------------------------
%
%\runsubsection{WPI}
%\descript{| \href{https://web.wpi.edu/Pubs/E-project/Available/E-project-042816-154409/unrestricted/MartinoManningFlaniganMQP.pdf}{\underline{Gamified Music Learning System with VR Force Feedback for Rehabilitation}}}
%
%\location{August 2015 - April 2016}
%Designed the mechanical and electrical systems for a custom haptic glove, utilizing a soft actuation system.  
%
%\sectionspace % Some whitespace after the section

%------------------------------------------------

%------------------------------------------------
% Awards
%------------------------------------------------

%\section{Awards} 
%
%\begin{tabular}{rll}
%2014	 & top 52/2500 & KPCB Engineering Fellow\\
%2014	 & 2\textsuperscript{nd} most points & Google Code Jam, Qualification Round\\
%2014	 & 1\textsuperscript{st}/50 & Microsoft Coding Competition, Cornell\\
%2013	 & National & Jump Trading Challenge Finalist\\
%2013 & 7\textsuperscript{th}/120 & CS 3410 Cache Race Bot Tournament \\
%2012 & 2\textsuperscript{nd}/150 & CS 3110 Biannual Intra-Class Bot Tournament \\
%2011 & National & Indian National Mathematics Olympiad (INMO) Finalist \\
%2010 & National & Comp. Soc. of India's National Programming Contest\\
%\end{tabular}
%
%\sectionspace % Some whitespace after the section
%
%----------------------------------------------------------------------------------------

% \end{minipage} % The end of the right column

%----------------------------------------------------------------------------------------
%	SECOND PAGE (EXAMPLE)
%----------------------------------------------------------------------------------------

%\newpage % Start a new page

%\begin{minipage}[t]{0.33\textwidth} % The left column takes up 33% of the text width of the page

%\section{Example Section}

%\end{minipage} % The end of the left column
%\hfill
%\begin{minipage}[t]{0.66\textwidth} % The right column takes up 66% of the text width of the page

%\section{Example Section 2}

%\end{minipage} % The end of the right column

%----------------------------------------------------------------------------------------

\end{document}